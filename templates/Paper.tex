% Diese Zeile ist ein Kommentar!
\documentclass[a4paper, 12pt]{scrarticle} % Dokumenttyp
% Angabe der genutzten Pakete
\usepackage[utf8]{inputenc} % Unterstützung für UTF-8
\usepackage[ngerman]{babel} % Spracheigenheiten
\usepackage{csquotes} % autom. Anführungzeichen
\usepackage[backend=biber, …]{biblatex}  % lit. Verzeichnis 
\usepackage{amsmath,amssymb} % math. Formelzeichen
\bibliography{Quellen.bib} % Angabe des Quellverzeichnisses

% Metadaten
\title{Einfache Vorlage eines Papers}
\author{Vorname Nachnahme, {et alias}}
\date{\today}

% Hauptteil
\begin{document}
\maketitle % Deckblatt oder Titel-Abschnitt
\tableofcontents % Inhaltsverzeichnis

\section{Abschnitt Überschrift}
Nachfolgend eine unsortierte Aufzählung wichtiger Strukturelemente in \LaTeX{}.
\begin{itemize}
	\item Zitate können mittels~\cite{bib-ref:1} referenziert werden.
	\item Text kann \textbf{fettgedruckt} oder \textit{kursiv} hervorgehoben werden, wobei eine \emph{Betonung} in der Regel reicht.
	\item (Sprachspezifische) \enquote{Anführungzeichen} können nach Einbindung des Pakets \texttt{csquotes} genutzt werden.
\end{itemize}
\newpage % Seitenumbruch

\subsection{Unterüberschrift}
Einfache Formeln wie $a = 5$, \(b = x^2\) und komplexere mittels
\begin{align*}
	\hat{\mathcal{L}}_2 &= \lim_{x \to 0} \sum^{\infty}_{k=1} \| \dot{\Theta}^{(k)} \|_x
\end{align*}

\printbibliography % Literaturverzeichnis
\end{document}
