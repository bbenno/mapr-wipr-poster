Die Setzung des in \LaTeX{} verfassten Quelltextes (\texttt{Paper.tex}) in ein pub\-li\-ka\-tions\-geeignetes Format, meist PDF, geschieht mittels verschiedenen Übersetzerprogrammen (\href{http://www.pdftex.org/}{\emph{pdfTeX}}, \href{http://xetex.sourceforge.net/}{\emph{XeTeX}} oder \href{http://www.luatex.org/}{\emph{LuaTex}}).
Diese sind Teil der offenen \href{https://tug.org/texlive/}{\emph{\TeX{} Live}-Distribution}, die noch weitere, grundlegende Pakete bereitstellt.
\vspace{0.5em}

Auch wenn zum Verfassen des Quelltextes ein einfacher Editor reicht, ist dieser Weg umständlich.
Abhilfe schaffen offene Entwicklungsumgebungen der Community: \href{https://www.xm1math.net/texmaker/}{\emph{Texmaker}} und \href{https://www.texstudio.org/}{\emph{TeXstudio}}.
Beide laufen auf allen etablierten Betriebssystemen.
\vspace{0.5em}

Für einen schnellen Einstieg ist die Online-Plattform \href{https://www.overleaf.com}{\emph{ Overleaf}} am geeignetsten, weil keine Einrichtung notwendig ist.
Zudem bietet diese eine einfachen Weg zur Kollaboration mehrerer Autoren an einem Paper.
