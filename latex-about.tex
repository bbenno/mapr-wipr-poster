\begin{wrapfigure}{R}{0.1\textwidth}
	\centering
	\includegraphics*[width=0.09\textwidth]{img/ctan-lion.png}
	\caption*{\hspace{7em}\color{lightgray}\small von Duane Bibby}
	\vspace{-6em}
\end{wrapfigure}

\large\LaTeX{} ist...
\begin{itemize}
	\item eine Programmiersprache für Textsetzung
	\item ein Werkzeugsatz aus Übersetzungsprogrammen
	\item ein Ecosystem aus Erweiterungen (sog. \emph{Pakete})
\end{itemize}

\vspace{0.05\linewidth}

\large\LaTeX{} bietet...
\begin{itemize}
	\item stilistische Einheitlichkeit
	\item ausgefeilte Strukturierungsmöglichkeiten
	\begin{itemize}
		\item Trennung von Inhalt und Design
		\item nicht-textuelle Inhalte (Strukturen aus der Mathematik, Chemie, Musik und mehr)
	\end{itemize}
	\item verschiedene Darstellungstypen: Folien, Poster {\color{gray}(wie dieses)} und weitere
	\item hochgradig erweiterbare Fähigkeiten
	\item eine steile Lernkurve, jedoch hohe Mächtigkeit und konsistente Ergebnisse
\end{itemize}

Motto: Der Nutzer kümmert sich um das Schreiben, \LaTeX{} kümmert sich um das Layout.
Zu vergleichende Ansätze:
\setlength{\columnsep}{1cm}
\begin{multicols}{2}
\centering

Microsoft\textsuperscript{\texttrademark} Word

\enquote{What you see is what you get.}

\columnbreak

\LaTeX{}

\enquote{You asked for it, you got it.}

\end{multicols}
